\documentclass[man,floatsintext]{apa6}
\usepackage{lmodern}
\usepackage{amssymb,amsmath}
\usepackage{ifxetex,ifluatex}
\usepackage{fixltx2e} % provides \textsubscript
\ifnum 0\ifxetex 1\fi\ifluatex 1\fi=0 % if pdftex
  \usepackage[T1]{fontenc}
  \usepackage[utf8]{inputenc}
\else % if luatex or xelatex
  \ifxetex
    \usepackage{mathspec}
  \else
    \usepackage{fontspec}
  \fi
  \defaultfontfeatures{Ligatures=TeX,Scale=MatchLowercase}
\fi
% use upquote if available, for straight quotes in verbatim environments
\IfFileExists{upquote.sty}{\usepackage{upquote}}{}
% use microtype if available
\IfFileExists{microtype.sty}{%
\usepackage{microtype}
\UseMicrotypeSet[protrusion]{basicmath} % disable protrusion for tt fonts
}{}
\usepackage{hyperref}
\hypersetup{unicode=true,
            pdftitle={REPRODUCED REPORT: REM sleep in naps differentially relates to memory consolidation in typical preschoolers and children with Down syndrome.},
            pdfauthor={Goffredina Spano~\& Rebecca L. Gomez},
            pdfkeywords={naps, sleep, memory, development, Down syndrome},
            pdfborder={0 0 0},
            breaklinks=true}
\urlstyle{same}  % don't use monospace font for urls
\usepackage{graphicx,grffile}
\makeatletter
\def\maxwidth{\ifdim\Gin@nat@width>\linewidth\linewidth\else\Gin@nat@width\fi}
\def\maxheight{\ifdim\Gin@nat@height>\textheight\textheight\else\Gin@nat@height\fi}
\makeatother
% Scale images if necessary, so that they will not overflow the page
% margins by default, and it is still possible to overwrite the defaults
% using explicit options in \includegraphics[width, height, ...]{}
\setkeys{Gin}{width=\maxwidth,height=\maxheight,keepaspectratio}
\IfFileExists{parskip.sty}{%
\usepackage{parskip}
}{% else
\setlength{\parindent}{0pt}
\setlength{\parskip}{6pt plus 2pt minus 1pt}
}
\setlength{\emergencystretch}{3em}  % prevent overfull lines
\providecommand{\tightlist}{%
  \setlength{\itemsep}{0pt}\setlength{\parskip}{0pt}}
\setcounter{secnumdepth}{0}
% Redefines (sub)paragraphs to behave more like sections
\ifx\paragraph\undefined\else
\let\oldparagraph\paragraph
\renewcommand{\paragraph}[1]{\oldparagraph{#1}\mbox{}}
\fi
\ifx\subparagraph\undefined\else
\let\oldsubparagraph\subparagraph
\renewcommand{\subparagraph}[1]{\oldsubparagraph{#1}\mbox{}}
\fi

%%% Use protect on footnotes to avoid problems with footnotes in titles
\let\rmarkdownfootnote\footnote%
\def\footnote{\protect\rmarkdownfootnote}


  \title{REPRODUCED REPORT: REM sleep in naps differentially relates to memory
consolidation in typical preschoolers and children with Down syndrome.}
    \author{Goffredina Spano\textsuperscript{1}~\& Rebecca L.
Gomez\textsuperscript{1,3}}
    \date{}
  
\shorttitle{REM sleep in preschoolers and children with Down syndrome}
\affiliation{
\vspace{0.5cm}
\textsuperscript{1} University of Arizona\\\textsuperscript{2} University College London}
\keywords{naps, sleep, memory, development, Down syndrome\newline\indent Word count: X}
\usepackage{csquotes}
\usepackage{upgreek}
\captionsetup{font=singlespacing,justification=justified}

\usepackage{longtable}
\usepackage{lscape}
\usepackage{multirow}
\usepackage{tabularx}
\usepackage[flushleft]{threeparttable}
\usepackage{threeparttablex}

\newenvironment{lltable}{\begin{landscape}\begin{center}\begin{ThreePartTable}}{\end{ThreePartTable}\end{center}\end{landscape}}

\makeatletter
\newcommand\LastLTentrywidth{1em}
\newlength\longtablewidth
\setlength{\longtablewidth}{1in}
\newcommand{\getlongtablewidth}{\begingroup \ifcsname LT@\roman{LT@tables}\endcsname \global\longtablewidth=0pt \renewcommand{\LT@entry}[2]{\global\advance\longtablewidth by ##2\relax\gdef\LastLTentrywidth{##2}}\@nameuse{LT@\roman{LT@tables}} \fi \endgroup}


\usepackage{lineno}

\linenumbers

\authornote{ This is a reproduction of an article published on
October 20, 2018.

Correspondence concerning this article should be addressed to Goffredina
Spano, Department of Psychology, University of Arizona, Tucson, AZ
85821. E-mail:
\href{mailto:g.spano@ucla.ac.uk}{\nolinkurl{g.spano@ucla.ac.uk}}}

\abstract{
Naps are beneficial for learning in typically developing infants,
children, and adults. They show greater retention when a delay between
training and test contains sleep then when it is a comparable period of
wake. However, individuals with Down syndrome have a high rate of
disordered sleep than seen in the typical population. Do they experience
the same benefits of sleep on learning? The current experiment suggests
they do not. While typically developing preschoolers showed more
retention after a period filled with sleep, children with Down syndrome
had greater retention after a period of wakefulness.


}

\begin{document}
\maketitle

\section{Methods}\label{methods}

\subsection{Participants}\label{participants}

\begin{verbatim}
## Warning: package 'xtable' was built under R version 3.5.3
\end{verbatim}

\begin{tabular}{l|r|r|r}
\hline
Groups & N & Mean\_age & PercentFemale\\
\hline
DS & 25 & 9.49 & 52\\
\hline
TD & 24 & 5.03 & 54\\
\hline
\end{tabular}

\subsection{Materials \& Procedure}\label{materials-procedure}

\begin{figure}
\centering
\includegraphics{C:/Users/mhorger/Documents/GitHub/Reproducible-Report_Horger/methods.PNG}
\caption{Methods}
\end{figure}

The goal of this study was to assess the retention of new words with
various intervals between training and test. Children received all
conditions 1-2 weeks apart. The conditions included: 1. after a 5 min
delay 2. after a nap (4 hour delay) 3. after 24 hours

\subsection{Data analysis}\label{data-analysis}

The authors assessed the number of trials needed to reach criterion
across conditions and groups.

The first analysis conducted was a repeated measures ANOVA for both wake
and nap conditions. The second was a 2x2 ANOVA with delay type as the
repeated factor and TD or DS as the between. These were conducted for
the 4 and 24 hour delay.

The second was a 2x2 ANOVA with delay type as the repeated factor and TD
or DS as the between. These were conducted for the 4 and 24 hour delay.

We used R (Version 3.5.2; R Core Team, 2018) and the R-packages
\emph{data.table} (Version 1.12.0; Dowle \& Srinivasan, 2019),
\emph{dplyr} (Version 0.8.0.1; Wickham, François, Henry, \& Müller,
2019), \emph{ggplot2} (Version 3.1.0; Wickham, 2016), \emph{papaja}
(Version 0.1.0.9842; Aust \& Barth, 2018), \emph{readxl} (Version 1.3.1;
Wickham \& Bryan, 2019), and \emph{xtable} (Version 1.8.3; Dahl, Scott,
Roosen, Magnusson, \& Swinton, 2018) for all our analyses.

\section{Results}\label{results}

\begin{tabular}{l|l|r|r}
\hline
Grouping & Timing & meanNTC & SEMNTC\\
\hline
DS & Immediate & 1.680000 & 0.2628054\\
\hline
DS & Sleep & 1.640000 & 0.1620699\\
\hline
DS & Wake & 2.080000 & 0.1993322\\
\hline
TD & Immediate & 2.041667 & 0.2789679\\
\hline
TD & Sleep & 1.708333 & 0.1408973\\
\hline
TD & Wake & 1.666667 & 0.2055980\\
\hline
\end{tabular}

\begin{figure}
\centering
\includegraphics{Midterm_in_papaja_files/figure-latex/NumberToCriterion-1.pdf}
\caption{\label{fig:NumberToCriterion}Average number of trials to criterion
per group per condition.}
\end{figure}

{[}1{]} \enquote{factor}

Error: Subjects Df Sum Sq Mean Sq F value Pr(\textgreater{}F) Residuals
48 2.206 0.04596

Error: Subjects:Condition Df Sum Sq Mean Sq F value Pr(\textgreater{}F)
Condition 1 0.007 0.00686 0.049 0.825 Residuals 48 6.674 0.13904

\begin{table}[tbp]
\begin{center}
\begin{threeparttable}
\caption{\label{tab:aovtable}ANOVA table for Experiment 1}
\begin{tabular}{lllllll}
\toprule
Effect & \multicolumn{1}{c}{$F$} & \multicolumn{1}{c}{$\mathit{df}_1$} & \multicolumn{1}{c}{$\mathit{df}_2$} & \multicolumn{1}{c}{$\mathit{MSE}$} & \multicolumn{1}{c}{$p$} & \multicolumn{1}{c}{$\hat{\eta}^2_G$}\\
\midrule
Condition & 0.05 & 1 & 48 & 0.14 & .825 & .001\\
\bottomrule
\end{tabular}
\end{threeparttable}
\end{center}
\end{table}

\section{Discussion}\label{discussion}

\newpage

\section{References}\label{references}

\begingroup
\setlength{\parindent}{-0.5in} \setlength{\leftskip}{0.5in}

\hypertarget{refs}{}
\hypertarget{ref-R-papaja}{}
Aust, F., \& Barth, M. (2018). \emph{papaja: Create APA manuscripts with
R Markdown}. Retrieved from \url{https://github.com/crsh/papaja}

\hypertarget{ref-R-xtable}{}
Dahl, D. B., Scott, D., Roosen, C., Magnusson, A., \& Swinton, J.
(2018). \emph{Xtable: Export tables to latex or html}. Retrieved from
\url{https://CRAN.R-project.org/package=xtable}

\hypertarget{ref-R-data.table}{}
Dowle, M., \& Srinivasan, A. (2019). \emph{Data.table: Extension of
`data.frame`}. Retrieved from
\url{https://CRAN.R-project.org/package=data.table}

\hypertarget{ref-R-base}{}
R Core Team. (2018). \emph{R: A language and environment for statistical
computing}. Vienna, Austria: R Foundation for Statistical Computing.
Retrieved from \url{https://www.R-project.org/}

\hypertarget{ref-R-ggplot2}{}
Wickham, H. (2016). \emph{Ggplot2: Elegant graphics for data analysis}.
Springer-Verlag New York. Retrieved from \url{http://ggplot2.org}

\hypertarget{ref-R-readxl}{}
Wickham, H., \& Bryan, J. (2019). \emph{Readxl: Read excel files}.
Retrieved from \url{https://CRAN.R-project.org/package=readxl}

\hypertarget{ref-R-dplyr}{}
Wickham, H., François, R., Henry, L., \& Müller, K. (2019). \emph{Dplyr:
A grammar of data manipulation}. Retrieved from
\url{https://CRAN.R-project.org/package=dplyr}

\endgroup


\end{document}
